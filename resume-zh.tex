%%
%% Copyright (c) 2018-2019 Weitian LI <wt@liwt.net>
%% CC BY 4.0 License
%%
%% Created: 2018-04-11
%%

% Chinese version
\documentclass[zh]{resume}
% Adjust icon size (default: same size as the text)
\iconsize{\Large}

% File information shown at the footer of the last page
\fileinfo{%
  \faCopyright{} 2021--2024, Zijian Wang \hspace{0.5em}
  \creativecommons{by}{4.0} \hspace{0.5em}
  \githublink{qumn}{resume} \hspace{0.5em}
  \faEdit{} \today
}

\name{子健}{汪}

\keywords{Linux, Programming, Java, JavaScript, Shell, Sql}

% \tagline{\icon{\faBinoculars}} <position-to-look-for>}
% \tagline{<current-position>}

% \photo{<height>}{<filename>}

\profile{
  \mobile{180-8661-7441}
  \email{qumn3497@qq.com}
  \icontext{\faFlask}{3年}
  \github{qumn} \\
  \university{武汉职业技术学院}
  \degree{软件工程}
  \birthday{2000-12-30}
  \address{武汉}
  % Custom information:
  % \icontext{<icon>}{<text>}
  % \iconlink{<icon>}{<link>}{<text>}
}

\begin{document}
\makeheader

%======================================================================
% Summary & Objectives
%======================================================================
{\onehalfspacing\hspace{2em}%
	计算机软件工程专业毕业生. 擅长 Java Web 开发,
	熟练掌握 Java, Rust 和 JavaScript 等编程语言, 热衷学习新技术, 在业余时间学习 Rust, Flutter 新兴等技术.
  有 4 年的 Linux 使用经验, 具有较强的运维能力. 喜欢使用 Linux 命令行工具, 如 Git, Tmux, NeoVim
	以及 \link{https://github.com/ibraheemdev/modern-unix}{Modern Unix} 等工具.
	\par}

%======================================================================
\sectionTitle{技能和语言}{\faWrench}
%======================================================================
\begin{competences}
	\comptence{操作系统}{%
		\icon{\faLinux} Linux (5 年)
	}
	\comptence{编程}{%
		Java, Kotlin, JavaScript, Rust, Sql, Dart
	}
	\comptence{工具}{%
		IDEA, Docker, SSH, Git, Httpie, Tmux, NeoVim
	}
	\comptence{后端框架}{%
		SpringBoot, SpringCloud Alibaba, Mybatis
	}
	\comptence{前端框架}{%
		Vue3, Jetpack Compose, Flutter, React
	}
	% \comptence{\icon{\faLanguage} 语言}{
	%   \textbf{英语} --- 读写(优良),听说(日常交流)
	% }
\end{competences}


%======================================================================
\sectionTitle{计算机技能}{\faCogs}
%======================================================================
\begin{itemize}
  \item 熟练掌握 Linux 常用命令, 具有丰富的 Linux 运维经验
	\item 熟悉 多线程并发编程, 熟悉 Java 并发包, 熟悉并发编程中数据竞争, 死锁等常见问题
	\item 熟悉 Java Web 开发, 掌握常用的 SpringBoot, SSM, MyBatis Plus 等常用开发框架
	\item 熟悉 Java 分布式常用开发框架 SpringCloud, OpenFeign, Nacos, Ribbon, Sentinel 等
	\item 了解 JVM 内部原理 类加载机制, 运行时内存布局, 垃圾回收机制等, 具有 Jvm 线上项目调优以及问题排查经验
  \item 熟悉 Mysql, PostgreSQL 等关系型数据库, 熟悉数据库性能优化
	\item 熟悉 Docker 熟悉常用的 Docker 命令, 熟悉 Docker Compose
	\item 熟悉常见算法和数据结构, \link{https://leetcode.cn/u/qu-ming-nan-o/}{\texttt{LeetCode}} 累计 120+ 题解
\end{itemize}

%======================================================================
\sectionTitle{工作经历}{\faBriefcase}
%======================================================================
\begin{experiences}
	\experience%
  [2022.07]%
  {至今}%
  {Java 工程师 @ {\color{accentcolor} 武汉中韬信息科技有限公司} }%
	[
    {\textbf{工作内容}}
      \begin{itemize}
        \item 搭建 Gitlab 私服, 实现前后端 CI/CD 持续集成
        \item 搭建 Zbox(禅道) 私服, 跟踪项目进度
        \item 编写前后端自动化构建发布脚本, 大幅度提升发版效率
        \item 编写自动化备份脚本, 备份数据库, 用户文件
        \item 负责智慧养老, 智慧工地项目性能优化.
        \item 引入 InfluxDb 时序数据库, 大幅度节省服务器资源
      \end{itemize}
  ]
	\separator{0.5ex}
	\experience%
  [2021.07]%
  {2022.06}%
  {Java 工程师 @ {\color{accentcolor} 武汉夏宇信息有限公司} }%
	[
    {\textbf{工作内容}}
      \begin{itemize}
        \item 负责线上项目服务器环境搭建以及维护
        \item 负责 统一管理平台, 机场运营管理 等项目开发
        \item 负责与新疆合作公司进行日常开发工作和维护工作
        \item 修改若依数据权限逻辑, 以实现跨数据库数据权限控制
      \end{itemize}
  ]

\end{experiences}

%======================================================================
\sectionTitle{教育背景}{\faGraduationCap}
%======================================================================
\begin{educations}
	\education%
	{2019.09}%
	[2022.06]%
	{武汉职业技术学院}%
	{软件技术}%
	{计算机科学}%
	{专科}
\end{educations}

\newpage
\sectionTitle{项目经历}{\faLaptop}

\begin{projects}
	\project
  {\color{accentcolor}{养老服务运营系统}}{2022.06 - 至今}
  {
    养老服务运营系统包含 App 端以及 Web 管理端, 各街道的网点在Web端中为老人下单上门养老服务.
    各个养老服务商的服务人员通过 App 抢单并上门服务. Web 端通过统计分析订单数据为养老专项资金下发提供凭证.
    \begin{itemize}
      \item 负责系统工作流多租户适配工作
      \item 通过责任链模式实现工单模块预警功能
      \item 通过 jstack 命令分析线上项目线程调用栈解决系统假死问题
      \item 服务器系统运维: 前后端自动化部署脚本, 数据备份脚本, 基础环境安装
      \item 通过缓存系统性能优化, 综合使用索引, 缓存, 线程池, 定时任务等方式实现
    \end{itemize}
  }
  {SpringBoot,Flowable,Mysql,Redis,MyBatis Plus,Mybatis Plus Join}

	\project
  {\color{accentcolor}{安全智慧工地管理平台}}{2023.11 - 至今}
  {
    安全智慧工地管理平台是基于物联网、大数据、AI技术,立足于施工现场的 "人、物、环境、管理" 四大安全监管要素,
    以构建智慧工地物联网硬件监测技术为主,以工地软件轻量化监管为辅,建立支撑现场管理、互联协同、智能决策、数据共享的信息化系统
    \begin{itemize}
      \item 引入 InfluxDb 时序数据库, 大幅度降低数据磁盘占用问题
      \item 对接地磅, 人脸识别, 塔吊监控等设实时数据
      \item 搭建项目运行环境, 使用 zx 编写后台自动化部署脚本
    \end{itemize}
  }
  {Spring Boot,Mqtt,Nacos,OpenFeign,MyBatis,Mysql,Redis,Golang}

	\project
  {\color{accentcolor}{智能养老社区服务}}{2023.06 - 2023.12}
  {
    智能养老社区服务是一个老人设备信息管理平台, 通过老人与设备关联, 通过设备实时数据对老人异常状态预警.
    当老人异常时联系工作人员进行处理.
    \begin{itemize}
      \item 设备管理模块的开发以及对接厂商数据
      \item 使用策略模式实现从 海康, 萤石, 电信等平台取流, 为前端提供统一的接口
      \item 扩展实现芋道权限系统, 实现基础数据无部门Id也可以使用权限系统
    \end{itemize}
  }
  {SpringBoot,MyBatis Plus,Mysql,Redis}

  \project
  {\color{accentcolor}{机场运营管理}}{2021.06 - 2022.07}
  {
    用于维护机场运营管理,包含航空公司管理、航班管理、学习资源管理、乘客管理等功能。项目使用采用分布式框架开发,
    基于 Ruoyi Cloud 框架. \\
    {\textbf{主要模块: } 航空公司管理、航班管理、大屏展示、学习资源管理、乘客管理、学习进度追踪、员工打卡统计} \\
    {\textbf{负责模块:} 大屏展示、学习进度追踪、员工打卡统计、维护项目优化部分接口的性能} \\
  }
  {Spring Cloud,Nacos,OpenFeign,MyBatis,Mysql,Redis,Vue,ElementUi}

	\project
  {\color{accentcolor}{中谷苑车辆道闸系统}}{2022.02 - 2022.07}
  {
    项目主要对小区业主车辆管理,临时车进行收费等
    \begin{itemize}
      \item 车辆管理是对小区业主的车辆进行月卡管理、月卡续费、地上下车库管理等
      \item 车辆监控是对月卡车辆、临时车进出记录的相关记录等
      \item 进出监管是基于小区内南北苑岗亭使用,使用websocket获取进出小区车辆数据来进行开道闸的操作
      \item 人员管理是针对岗亭和财务人员的管理
      \item 收入对账是财务人员查询岗亭的临时车收费或者月卡的月卡续费等数据统计
    \end{itemize}
  }
  {Nacos,Spring Cloud,SpringBoot,MyBatis,Mysql,Vue,ElementUi,Redis}

	\project
  {\color{accentcolor}{田园网}}{2021.09 - 2022.01}
  {
    用于海南省农产品销售平台,用户农户的产品种植计划、商户注册、个类农产品的统计。项目使用采用分 布式框架开发,
    技术采用Spring Boot、Spring Cloud \& Alibaba注册中心、配置中心选型Nacos,流量控制框架选 型Sentinel。
    采用前后端分离,包含微信小程序端
    \begin{itemize}
      \item 田园网商品、商户展示
      \item 订单管理商家可以对订单进行管理(查询, 接单/拒单, 统计, 生成报表)
      \item 商品分类管理
      \item 商家信息填写审核以及商家上架(使用版本号控制后台审核以及门户的商户信息,审核通过同步最新商户信息)
    \end{itemize}
  }
  {Nacos,Spring Cloud,SpringBoot,MyBatis,Mysql,ElementUi,Redis}

	\project
  {\color{accentcolor}{统一管理平台}}{2021.06 - 2021.09}
  {
    公司内部管理存在多个管理系统, 为了方便用户使用, 开发了一个统一管理平台, 
    通过此平台用户可以一次登录访问其他项目而且同意管理其他项目的权限和字典等模块. 
    因为需要在一个项目中管理多个其他项目, 需要使用多数据源, 并且通过AOP根据不同的请求参数切换数据源
    \begin{itemize}
      \item 统一登录在此项目中登录后可以直接访问其他项目无需另外登录。
      \item 字典管理是对其他项目的字典进行统一管理。
      \item 权限管理是对其他项目的权限进行统一管理。
    \end{itemize}
  }
  {SpringBoot,多数据源,Redis}


\end{projects}

\sectionTitle{个人项目}{\faCode}
\begin{projects}
	\project
  {\color{accentcolor}{alumnihub}}{2024.02 - 至今}
  {
    \website{https://github.com/qumn/alumnihub}{alumnihub}校园生活为个人学习DDD(领域驱动设计)和CQRS(命令查询责任分离)思想的实践项目. \\
    项目分为 Comment, Trade, Forum, LostFound 以及 User 5个 Bounded Contex(界限上下文)
    \begin{itemize}
      \item App 端使用 Jetpack Compose 开发
      \item User 与其他业务 Bounded Context 为 Shared Kernal 模式
      \item Comment 与 Trade, Forum, LostFound 为 Customer/Supplier 模式
      \item 采用 testcontainers 实现数据库集成测试
      \item ORM 采用 Ktorm 自定义连接器实现逻辑删除(支持链接查询, 子查询)
    \end{itemize}
  }
  {Kotlin,gradle,ktorm,Axon,MQTT,PostgreSQL,SpringBoot,SpringSecurity,Jetpack Compose}

	\project
  {\color{accentcolor}{jvmrs}}{2022.01 - 2022.02}
  {
    \website{https://github.com/qumn/jvmrs}{jvmrs}个人学习项目, 使用Rust简单的实现JVM虚拟机, 用于学习Rust语言. \\
    {\textbf{预计模块: } 类加载器、运行时数据区、解析类文件、指令和解释器、方法调用和返回、本体方法调用、异常处理、垃圾回收器。} \\
    {\textbf{完成模块:} 类加载器、运行时数据区、指令和解释器。} \\
  }
  {Rust,bytes,clap,anyhow}

\end{projects}

\end{document}
