%%
%% Copyright (c) 2018-2019 Weitian LI <wt@liwt.net>
%% CC BY 4.0 License
%%
%% Created: 2018-04-11
%%

% Chinese version
\documentclass[zh]{resume}
% Adjust icon size (default: same size as the text)
\iconsize{\Large}

% File information shown at the footer of the last page
\fileinfo{%
  \faCopyright{} 2018--2020, Zijian Wang \hspace{0.5em}
  \creativecommons{by}{4.0} \hspace{0.5em}
  \githublink{qumn}{resume} \hspace{0.5em}
  \faEdit{} \today
}

\name{子健}{汪}

\keywords{Linux, Programming, Java, JavaScript, Shell, Sql}

% \tagline{\icon{\faBinoculars}} <position-to-look-for>}
% \tagline{<current-position>}

% \photo{<height>}{<filename>}

\profile{
  \mobile{180-8661-7441}
  \email{qumn3497@qq.com}
  \icontext{\faFlask}{2年}
  \github{qumn} \\
  \university{武汉职业技术学院}
  \degree{软件工程 \textbullet 大专}
  \birthday{2000-12-30}
  \address{武汉}
  % Custom information:
  % \icontext{<icon>}{<text>}
  % \iconlink{<icon>}{<link>}{<text>}
}

\begin{document}
\makeheader

%======================================================================
% Summary & Objectives
%======================================================================
{\onehalfspacing\hspace{2em}%
	计算机软件工程专业毕业生, 擅长 Java Web 开发,
	有 4 年的 Linux 使用经验,熟练掌握 Java, Rust 和 JavaScript 等编程语言.
	热衷学习新技术, 在业余时间学习 Rust, Flutter 新兴等技术.
	喜欢使用 Linux 命令行工具, 如 Git, Tmux, NeoVim
	以及 \link{https://github.com/ibraheemdev/modern-unix}{Modern Unix} 等工具.
	\par}

%======================================================================
\sectionTitle{技能和语言}{\faWrench}
%======================================================================
\begin{competences}
	\comptence{操作系统}{%
		\icon{\faLinux} Linux (4 年)
	}
	\comptence{编程}{%
		Java, JavaScript, Rust, Sql, Dart
	}
	\comptence{工具}{%
		IDEA, Docker, SSH, Git, Tmux, NeoVim
	}
	\comptence{前端开发}{%
		Vue3, Flutter, React
	}
	\comptence{后端开发}{%
		SpringBoot, SpringCloud Alibaba, Mybatis
	}
	% \comptence{\icon{\faLanguage} 语言}{
	%   \textbf{英语} --- 读写(优良),听说(日常交流)
	% }
\end{competences}

%======================================================================
\sectionTitle{教育背景}{\faGraduationCap}
%======================================================================
\begin{educations}
	\education%
	{2022.06}%
	[2019.09]%
	{武汉职业技术学院}%
	{软件技术}%
	{计算机科学}%
	{大专}
\end{educations}

%======================================================================
\sectionTitle{计算机技能}{\faCogs}
%======================================================================
\begin{itemize}
	\item 熟悉 Java Web 开发, 掌握常用的 SpringBoot, SSM, MyBatisPlus 等框架.
	\item 熟悉 SpringCloud Alibaba, 熟练使用常用的 Nacos, Gateway, Ribbon, Sentinel 等组件
	\item 熟悉 Vue3, Flutter, React 等前端开发框架
	\item 熟悉 Java, Rust, JavaScript, Dart 等编程语言
	\item 熟悉关系型数据库 MySQL, 熟悉 Redis 等 NoSQL 数据库
	\item 了解 JVM 内部原理, 如: 类加载机制, 运行时内存布局, 垃圾回收机制等
	\item 熟悉 多线程并发编程, 熟悉 Java 并发包, 熟悉常见的锁机制
	\item 熟悉 Linux 操作系统, 熟练使用 Git, Tmux, NeoVim 等工具
	\item 熟悉常见算法和数据结构, \link{https://leetcode.cn/u/qu-ming-nan-o/}{\texttt{LeetCode}} 累计 120+ 题解
	\item 熟悉Docker, 熟悉常用的 Docker 命令, 熟悉 Docker Compose
\end{itemize}

%======================================================================
\sectionTitle{工作经历}{\faBriefcase}
%======================================================================
\begin{experiences}
	\experience%
  [2021.06]%
  {至今}%
	{开发工程师 @ 武汉夏宇信息有限公司 }%
	[
    {\textbf{工作内容}}
      \begin{itemize}
        \item 负责与新疆合作公司进行日常开发工作(对接需求,开需求评审会议,然后根据原型图进行开发)和维护工作(解决禅道 BUG)
        \item 负责公司的项目,进行相关分配的模块开发
        \item 研究项目框架优化以及合理搭配中间件
      \end{itemize}
    {\textbf{业绩}}
      \begin{itemize}
        \item 参与新疆凯亚产品研发工作以及日常维护,主要针对于机场大屏展示, 学习进度追踪等
        \item 参与田园网项目开发
        \item 参与武汉高铁工务段仓库管理系统开发
        \item 参与装备保障项目模块开发,文档编写
        \item 负责智慧农业模块开发以及硬件对接
      \end{itemize}
  ]

	\separator{0.5ex}
\end{experiences}

\newpage
\sectionTitle{项目经历}{\faLaptop}

\begin{projects}
	\project
  {\color{accentcolor}{机场运营管理}}{2022.02 - 2022.06}
  {
    用于维护机场运营管理,包含航空公司管理、航班管理、学习资源管理、乘客管理等功能。项目使用采用分布式框架开发,
    基于 Ruoyi Cloud 框架. \\
  {\textbf{主要模块: } 航空公司管理、航班管理、大屏展示、学习资源管理、乘客管理、学习进度追踪、员工打卡统计。} \\
  {\textbf{负责模块:} 大屏展示、学习进度追踪、员工打卡统计、维护项目优化部分接口的性能。}
  }
  {Nacos,Spring Cloud,SpringBoot,MyBatis,Mysql,Vue,ElementUi,Redis}

	\project
  {\color{accentcolor}{田园网}}{2021.08 - 2022.02}
  {
    用于海南省农产品销售平台,用户农户的产品种植计划、商户注册、个类农产品的统计。项目使用采用分 布式框架开发,
    技术采用Spring Boot、Spring Cloud \& Alibaba注册中心、配置中心选型Nacos,流量控制框架选 型Sentinel。
    采用前后端分离,包含微信小程序端。\\ 
  {\textbf{主要模块: } 门户商品展示、个人中心、后台商品发布等。} \\
  {\textbf{负责模块:} 门户商品展示,商家管理(商家审核、商家上下架)。}
  }
  {Nacos,Spring Cloud,SpringBoot,MyBatis,Mysql,ElementUi,Redis}

	\project
  {\color{accentcolor}{统一管理平台}}{2021.06 - 2021.08}
  {
    公司内部管理存在多个管理系统, 为了方便用户使用, 开发了一个统一管理平台, 
    通过此平台用户可以一次登录访问其他项目而且同意管理其他项目的权限和字典等模块. 
    因为需要在一个项目中管理多个其他项目, 需要使用多数据源, 并且通过AOP根据不同的请求参数切换数据源\\
    {\textbf{主要模块: } 统一登录、字典管理、权限管理。} \\
    {\textbf{负责模块:} 统一登录、字典管理、权限管理。}
  }
  {SpringBoot,多数据源,Redis}

	\project
  {\color{accentcolor}{武汉高铁工务段仓库APP}}{2021.06 - 2021.08}
  {
    武汉高铁工务段仓库物质的管理,上下架、盘点等操作。使用UniApp开发的App程序
    采用前后端分离,包含微信小程序端。\\ 
    {\textbf{主要模块: } 物料上架、物料下架、仓库盘点。} \\
    {\textbf{负责模块:} 物料上架、物料下架、仓库盘点。}
  }
  {Vue,UniApp,VueX}


	\project
  {\color{accentcolor}{LocalBiz}}{2021.06 - 2021.08}
  {
    \website{https://github.com/qumn/localBiz}{LocalBiz}意为 Local Business 附近的商业,  此项目旨在使用户可以足不出户的体验附近的商业资源。
    项目前端使用 Flutter 开发, 提供跨平台的支持. 后端在Ruoyi Cloud的基础上开发 \\
    {\textbf{主要模块: } 商家管理、商品管理、订单管理、用户管理等。} \\
    {\textbf{负责模块:} 商家管理、商品管理、订单管理、用户管理等。}
  }
  {Flutter,Nacos,Spring Cloud,SpringBoot,MyBatis,Mysql,NaiveUI,Redis}



	\project
  {\color{accentcolor}{jvmrs}}{2022.01 - 2022.02}
  {
    \website{https://github.com/qumn/jvmrs}{jvmrs}个人学习项目, 使用Rust简单的实现JVM虚拟机, 用于学习Rust语言. \\
    {\textbf{预计模块: } 类加载器、运行时数据区、解析类文件、指令和解释器、方法调用和返回、本体方法调用、异常处理、垃圾回收器。} \\
    {\textbf{完成模块:} 类加载器、运行时数据区、指令和解释器。}
  }
  {Rust,bytes,clap,anyhow}

	\project
  {\color{accentcolor}{eChat}}{2022.01 - 2022.02}
  {
    \website{https://github.com/qumn/eChat}{eChat}个人学习项目, 使用Rust简单实现IM聊天的后台.\\
    用于学习WebSocket, Rust语言, 协程, 网络通信协议. \\
    {\textbf{预计模块: } 消息通信, 好友管理, 聊天群管理} \\
    {\textbf{完成模块:} 消息通信, 好友管理, 聊天群管理}
  }
  {Rust,tower,tokio,sqlx,chrono,serde}

\end{projects}

\end{document}
