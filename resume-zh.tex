%%
%% Copyright (c) 2018-2019 Weitian LI <wt@liwt.net>
%% CC BY 4.0 License
%%
%% Created: 2018-04-11
%%

% Chinese version
\documentclass[zh]{resume}
% Adjust icon size (default: same size as the text)
\iconsize{\Large}

% File information shown at the footer of the last page
\fileinfo{%
  \faCopyright{} 2018--2020, Zijian Wang \hspace{0.5em}
  \creativecommons{by}{4.0} \hspace{0.5em}
  \githublink{qumn}{resume} \hspace{0.5em}
  \faEdit{} \today
}

\name{子健}{汪}

\keywords{Linux, Programming, Java, JavaScript, Shell, Sql}

% \tagline{\icon{\faBinoculars}} <position-to-look-for>}
% \tagline{<current-position>}

% \photo{<height>}{<filename>}

\profile{
  \mobile{180-8661-7441}
  \email{qumn3497@qq.com}
  \icontext{\faFlask}{2年}
  \github{qumn} \\
  \university{武汉职业技术学院}
  \degree{软件工程 \textbullet 大专}
  \birthday{2000-12-30}
  \address{武汉}
  % Custom information:
  % \icontext{<icon>}{<text>}
  % \iconlink{<icon>}{<link>}{<text>}
}

\begin{document}
\makeheader

%======================================================================
% Summary & Objectives
%======================================================================
{\onehalfspacing\hspace{2em}%
	计算机软件工程专业毕业生, 擅长 Java Web 开发,
	有 4 年的 Linux 使用经验,熟练掌握 Java, Rust 和 JavaScript 等编程语言.
	热衷学习新技术, 在业余时间学习了 Rust, Flutter 新兴等技术.
	喜欢使用 Linux 命令行工具, 如 Git, Tmux, NeoVim
	以及 \link{https://github.com/ibraheemdev/modern-unix}{Modern Unix} 等工具.
	\par}

%======================================================================
\sectionTitle{技能和语言}{\faWrench}
%======================================================================
\begin{competences}
	\comptence{操作系统}{%
		\icon{\faLinux} Linux (4 年)
	}
	\comptence{编程}{%
		Java, JavaScript, Rust, Sql, Dart
	}
	\comptence{工具}{%
		IDEA, Docker, SSH, Git, Tmux, NeoVim
	}
	\comptence{前端开发}{%
		Vue3, Flutter, React
	}
	\comptence{后端开发}{%
		SpringBoot, SpringCloud Alibaba, Mybatis Plus
	}
	% \comptence{\icon{\faLanguage} 语言}{
	%   \textbf{英语} --- 读写(优良),听说(日常交流)
	% }
\end{competences}

%======================================================================
\sectionTitle{教育背景}{\faGraduationCap}
%======================================================================
\begin{educations}
	\education%
	{2022.06}%
	[2019.09]%
	{武汉职业技术学院}%
	{软件技术}%
	{计算机科学}%
	{大专}
\end{educations}

%======================================================================
\sectionTitle{计算机技能}{\faCogs}
%======================================================================
\begin{itemize}
	\item 熟悉 Java Web 开发, 掌握常用的 SpringBoot, SSM, MyBatisPlus 等框架.
	\item 熟悉 SpringCloud Alibaba, 熟练使用常用的 Nacos, Gateway, Ribbon, Sentinel 等组件
	\item 熟悉 Vue3, Flutter, React 等前端开发框架
	\item 熟悉 Java, Rust, JavaScript, Dart 等编程语言
	\item 熟悉关系型数据库 MySQL, 熟悉 Redis, MongoDB 等 NoSQL 数据库
	\item 了解 JVM 内部原理, 如: 类加载机制, 运行时内存布局, 垃圾回收机制等
	\item 熟悉 Linux 操作系统, 熟练使用 Git, Tmux, NeoVim 等工具
	\item 熟悉常见算法和数据结构, \link{https://leetcode.cn/u/qu-ming-nan-o/}{\texttt{LeetCode}} 累计 120+ 题解
	\item 在校期间负责维护学校工作室的服务器, 为学校提供网站建设服务
\end{itemize}

%======================================================================
\sectionTitle{个人项目}{\faCode}
%======================================================================
\begin{itemize}
	\item \link{https://github.com/qumn/LocalBiz}{\texttt{LocalBiz}}:
	      本地商业服务平台, 为用户提供附近的商业服务信息
	      \begin{itemize}
		      \item \link{https://github.com/qumn/LocalBizFlutter}{\underline{\texttt{{\faGithub}客户端}}} 使用 Flutter 开发, 具有跨平台的特性支持 Android, iOS, Web, Linux, Mac 和 Windows
		      \item \link{https://github.com/qumn/LocalBiz}{\underline{\texttt{{\faGithub}后端}}} 基于Ruoyi Cloud开发, 使用SpringBoot, SpringCloud Alibaba, Mybatis Plus等技术
		      \item ORM 测试 \link{https://github.com/database-rider/database-rider}{\texttt{DataBase Rider}} 框架进行测试
	      \end{itemize}
	\item \link{https://gitee.com/qumn/cotm}{\texttt{cotm}}:
	      跨组织人才管理系统, 为企业提供人才管理服务
	      \begin{itemize}
		      \item \link{https://gitee.com/qumn/cotm-ui}{\underline{\texttt{{\faGithub}前端}}}: Vue, VueX, ElementUI和Axios 等
		      \item \link{https://gitee.com/qumn/cotm}{\underline{\texttt{{\faGithub}后端}}}: SpringBoot, Shiro, MyBatisPlus, Redis 等
	      \end{itemize}
	\item \link{https://github.com/qumn/jvmrs}{\underline{\texttt{{\faGithub}jvmrs}}}:
	      用 Rust 编写的 JVM, 用于学习 JVM 原理
	      \begin{itemize}
		      \item Class 文件解析
		      \item 运行时数据区
		      \item 大部分字节码解释器
	      \end{itemize}
	\item \link{https://github.com/qumn/eChat/blob/main/src/main.rs}{\underline{\texttt{{\faGithub}eChat}}}:
	      用 Rust 编写的聊天室, 用于学习 Rust, 以及{\textbf{协程}}.
	      \begin{itemize}
		      \item Web 框架 \link{https://github.com/tokio-rs/axum}{\texttt{axum}}.
		      \item 异步运行时: \link{https://github.com/tokio-rs/tokio}{\texttt{tokio}}
		      \item 数据库: \link{https://github.com/launchbadge/sqlx}{\texttt{sqlx}}
	      \end{itemize}
\end{itemize}

\newpage
%======================================================================
\sectionTitle{项目经历}{\faBriefcase}
%======================================================================
\begin{experiences}
	\experience%
	[2022.04]%
  {2021.07}%
	{开发工程师 @ 武汉夏宇信息有限公司 }%
	[\begin{itemize}
			\item {\textbf{统一登录平台}}\\
			      为公司内部基于不同框架开发的多个系统提供统一的登录平台
			      \begin{itemize}
				      \item 前端: 使用Vue3开发的单页面应用
				      \item 使用 Ruoyi 开发的项目, 提供统一的权限管理
				      \item 和多个系统进行对接, 使用JWT 或 Cookie 进行身份认证
			      \end{itemize}
			\item {\textbf{航空管理系统}}\\
			      包含航空公司管理, 航班管理, 学习资源管理, 乘客管理等功能 \\
			      项目职责
			      \begin{itemize}
				      \item 大屏展示接口开发
				      \item 内部员工培训学习资源管理模块开发
				      \item 维护已经存在的接口
			      \end{itemize}
		\end{itemize}]
	\separator{0.5ex}

	\experience%
	[2023.06]%
	{2022.04}%
	{开发工程师 @ 华通科技 }%
	[\begin{itemize}
			\item {\textbf{国铁物资系统}}\\
			      一个仓库库存管理系统, 具有库存管理, 库存预警, 库存盘点等功能
			      \begin{itemize}
				      \item 前端: 使用uni-app开发的微信小程序
				      \item 后端: SpringBoot, Shiro, MyBatisPlus, Redis 等
			      \end{itemize}
			      项目职责
			      \begin{itemize}
              \item 负责物资管理模块的开发
              \item 负责库存管理模块的开发
              \item 负责库存入库模块的开发
			      \end{itemize}
			\item {\textbf{装备流程管理系统}} \\
			      此项目包含装备组装流程, 装备审核流程, 装备出库流程等功能.
			      \begin{itemize}
				      \item 前端: 基于\link{https://github.com/layui/layui}{\underline{\texttt{layui}}} 框架开发
				      \item 后端: 基于\link{https://gitee.com/stylefeng/guns}{\underline{\texttt{guns}}} 框架开发, 有 SpringBoot, MyBatis, Shiro, Redis 等.
			      \end{itemize}
			      项目职责
			      \begin{itemize}
				      \item 开发与配置,产品项目显示化流程需求对接
				      \item 负责装备审核流程管理模块的后端以及前端的开发
				      \item 负责编写Dockerfile, 以及docker-compose文件部署项目
			      \end{itemize}
		\end{itemize}]
	\separator{0.5ex}

\end{experiences}
\end{document}

